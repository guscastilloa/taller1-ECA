\documentclass[12pt, letter]{article}
\usepackage[utf8]{inputenc}
\usepackage[spanish]{babel}
\usepackage{amsmath, amssymb, amsthm}
\usepackage{graphicx}
\usepackage{float}
\usepackage{hyperref}
\usepackage{booktabs}
\usepackage{geometry}
\geometry{
 letterpaper,
 total={170mm,257mm},
 left=20mm,
 top=20mm,
 bottom = 20mm
 }
\usepackage{enumitem}



\title{Solución del Taller 1 \\ \large Econometría Avanzada 2025-1}
\author{Ana María Patrón \\
 Gustavo A. Castillo Alvarez}
\date{\today}

\begin{document}

\maketitle

\tableofcontents
\newpage



%%%%%%%%%%%%%%%%%%%%%%%%%%%%%%%%%%%%%%%%%%%
% ▗▄▄▖  ▗▄▖ ▗▄▄▖▗▄▄▄▖▗▄▄▄▖     ▗▄▖ 
% ▐▌ ▐▌▐▌ ▐▌▐▌ ▐▌ █  ▐▌       ▐▌ ▐▌
% ▐▛▀▘ ▐▛▀▜▌▐▛▀▚▖ █  ▐▛▀▀▘    ▐▛▀▜▌
% ▐▌   ▐▌ ▐▌▐▌ ▐▌ █  ▐▙▄▄▖    ▐▌ ▐▌
%%%%%%%%%%%%%%%%%%%%%%%%%%%%%%%%%%%%%%%%%%%

\section{Parte A: El Impacto de Publicar en Temas de Investigación de Tendencia}
\subsection*{A1. Describa el problema de estimación usando el lenguaje de resultados potenciales.}

Sabemos que $Y_i$ representa el número de citaciones obtenidas en el corte de datos por el investigador $i$. Asimismo tenemos que 
$$
D_i=\begin{cases}
        1, & \text{Si $i$ publica en \textit{trending topic}}\\
        0, & \text{de lo contrario}
    \end{cases}
$$



\subsection{A2. Representación matemática de mecanismo de asignación}



\subsection{A3. }
\subsection{A4. Demostración consistencia estimador}
\subsection{A5. Análisis e interpretación ATE, ATT, ATU}

\begin{enumerate}[label=A.\arabic*]
    \item Describa el problema de estimación usando el lenguaje de resultados potenciales.
    \begin{enumerate}[label=A.1.\alph*]
        \item Describa los resultados potenciales en contexto.
        \item Explique el problema de inferencia causal.
        \item Escriba e interprete el ATE, ATT y ATU.
    \end{enumerate}


    \item Representación matemática de los mecanismos de asignación.
    \begin{enumerate}[label=A.2.\alph*]
        \item ¿Es aleatoria la Alternativa 1? Justifique (150 palabras).
        \item ¿Es aleatoria la Alternativa 2? Justifique (150 palabras).
        \item Discuta problemas de inferencia causal si la asignación no es aleatoria (150 palabras).
    \end{enumerate}
    \item Demostración de la consistencia del estimador.
    \item Interpretación y análisis de los parámetros ATT, ATU, y ATE.
\end{enumerate}







%%%%%%%%%%%%%%%%%%%%%%%%%%%%%%%%%%%%%%%%%%%
% ▗▄▄▖  ▗▄▖ ▗▄▄▖▗▄▄▄▖▗▄▄▄▖    ▗▄▄▖ 
% ▐▌ ▐▌▐▌ ▐▌▐▌ ▐▌ █  ▐▌       ▐▌ ▐▌
% ▐▛▀▘ ▐▛▀▜▌▐▛▀▚▖ █  ▐▛▀▀▘    ▐▛▀▚▖
% ▐▌   ▐▌ ▐▌▐▌ ▐▌ █  ▐▙▄▄▖    ▐▙▄▞▘
%%%%%%%%%%%%%%%%%%%%%%%%%%%%%%%%%%%%%%%%%%%
                                 
                                
\section{Parte B: Teorema de Frisch-Waugh-Lovell}
\subsection{B1. Demostraciones relacionadas con proyecciones y residuales.}

Primero veamos que la matriz $\textbf{P} = \textbf{X}(\textbf{X}'\textbf{X})^{-1}\textbf{X}'$ es una proyección ortogonal. Para ello debemos ver si esta es idempotente y simétrica. En exte ejercicio utilizaremos las mayúsculas $X$ como matrices para facilitar la notación. 



\( P = X (X'X)^{-1} X' \), \quad \( X \in \mathbb{R}^{n \times k} \), \quad \( P \in \mathbb{R}^{n \times n} \)

\subsubsection*{Idempotencia:}
\begin{align*}
P P &= \left( X (X'X)^{-1} X' \right) \left( X (X'X)^{-1} X' \right) \\
    &= X (X'X)^{-1} (X'X) (X'X)^{-1} X' \\
    &= X (X'X)^{-1} X' \\
    &= P
\end{align*}

\subsubsection*{Simetría:}
\begin{align*}
P' &= \left( X (X'X)^{-1} X' \right)' \\
   &= \left(  \right) \\
   &= \left( X' \right)' \left( (X'X)^{-1} \right)' X' \\
   &= X \left( (X'X)^{-1} \right)' X' \\
   &= X \left( (X'X)' \right)^{-1} X' \\
   &= X \left( X'X \right)^{-1} X' \\
   &= P
\end{align*}


\subsection{B2. Regresión particionada: derivación y demostración.}







%%%%%%%%%%%%%%%%%%%%%%%%%%%%%%%%%%%%%%%%%%%
% ▗▄▄▖  ▗▄▖ ▗▄▄▖▗▄▄▄▖▗▄▄▄▖     ▗▄▄▖
% ▐▌ ▐▌▐▌ ▐▌▐▌ ▐▌ █  ▐▌       ▐▌   
% ▐▛▀▘ ▐▛▀▜▌▐▛▀▚▖ █  ▐▛▀▀▘    ▐▌   
% ▐▌   ▐▌ ▐▌▐▌ ▐▌ █  ▐▙▄▄▖    ▝▚▄▄▖
%%%%%%%%%%%%%%%%%%%%%%%%%%%%%%%%%%%%%%%%%%%

\section{Parte C: Aplicación Práctica — Conflicto en Darfur}
\begin{enumerate}[label=C.\arabic*]
    \item Discusión del problema de identificación causal.
    \item Uso del Teorema de Frisch-Waugh-Lovell para análisis de sesgo.
\end{enumerate}




\section{Parte D: Diseño Experimental en Punjab}
\subsection{Configuración del Experimento y Análisis}
\begin{enumerate}[label=D.\arabic*]
    \item Definición de asignación y resultados potenciales.
    \item Balance de muestra y especificación de efectos de tratamiento.
\end{enumerate}

%%%%%%%%%%%%%%%%%%%%%%%%%%%%%%%%%%%%%%%%%%%
% ▗▄▄▖  ▗▄▖ ▗▄▄▖▗▄▄▄▖▗▄▄▄▖    ▗▄▄▄ 
% ▐▌ ▐▌▐▌ ▐▌▐▌ ▐▌ █  ▐▌       ▐▌  █
% ▐▛▀▘ ▐▛▀▜▌▐▛▀▚▖ █  ▐▛▀▀▘    ▐▌  █
% ▐▌   ▐▌ ▐▌▐▌ ▐▌ █  ▐▙▄▄▖    ▐▙▄▄▀
%%%%%%%%%%%%%%%%%%%%%%%%%%%%%%%%%%%%%%%%%%%
\subsection{Estimación y Análisis de Regresión}
\begin{enumerate}[label=D.\arabic*]
    \item Estimación con controles y efectos fijos.
    \item Análisis del comportamiento oportunista y su impacto en los resultados.
\end{enumerate}

\section*{Conclusión}
\noindent En este documento se resolvieron los ejercicios planteados, abordando problemas de identificación causal, análisis de regresión y diseño experimental.

\end{document}
