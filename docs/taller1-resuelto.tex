\documentclass[12pt, letter]{article}
\usepackage[utf8]{inputenc}
\usepackage[spanish]{babel}
\usepackage{amsmath, amssymb, amsthm}
\newtheorem{definition}{Definición}
\newtheorem{theorem}{Teorema}
\usepackage{algorithm}
\usepackage{algorithmic}
\usepackage{algpseudocode}
\usepackage{xcolor}
\usepackage{graphicx}
\usepackage{float}
\usepackage{hyperref}
\usepackage{booktabs}
\usepackage{geometry}
\geometry{
 letterpaper,
 total={170mm,257mm},
 left=20mm,
 top=20mm,
 bottom = 20mm
 }
\usepackage{enumitem}



\title{Solución del Taller 1 \\ \large Econometría Avanzada 2025-1}
\author{Ana María Patrón \\
 Gustavo A. Castillo Alvarez}
\date{\today}

\begin{document}

\maketitle

\tableofcontents
\newpage



%%%%%%%%%%%%%%%%%%%%%%%%%%%%%%%%%%%%%%%%%%%
% ▗▄▄▖  ▗▄▖ ▗▄▄▖▗▄▄▄▖▗▄▄▄▖     ▗▄▖ 
% ▐▌ ▐▌▐▌ ▐▌▐▌ ▐▌ █  ▐▌       ▐▌ ▐▌
% ▐▛▀▘ ▐▛▀▜▌▐▛▀▚▖ █  ▐▛▀▀▘    ▐▛▀▜▌
% ▐▌   ▐▌ ▐▌▐▌ ▐▌ █  ▐▙▄▄▖    ▐▌ ▐▌
%%%%%%%%%%%%%%%%%%%%%%%%%%%%%%%%%%%%%%%%%%%

\section{Parte A: El Impacto de Publicar en Temas de Investigación de Tendencia}
\subsection*{A1. Describa el problema de estimación usando el lenguaje de resultados potenciales.}

Sabemos que $Y_i$ representa el número de citaciones obtenidas en el corte de datos por el investigador $i$. Asimismo tenemos que 
$$
D_i=\begin{cases}
        1, & \text{Si $i$ publica en \textit{trending topic}}\\
        0, & \text{de lo contrario}
    \end{cases}
$$

El problema de estimación se puede describir en términos de resultados potenciales de la siguiente manera:

\begin{align*}
    Y_i(1) &= \text{Número de citaciones si el investigador $i$ publica en \textit{trending topic}} \\
    Y_i(0) &= \text{Número de citaciones si el investigador $i$ no publica en \textit{trending topic}}
\end{align*}

Sin embargo, como en la realidad solo observamos uno de los resultados potenciales, podemos escribir el número de citaciones obtenidas por el investigador $i$ como:

$$
Y_i = Y_i(1)D_i + Y_i(0)(1-D_i)
$$

Por lo tanto, definimos el efecto de tratamiento individual (ITE) como:

$$
\tau_i = Y_i(1) - Y_i(0)
$$

Como nos interesa en realidad es la población y no solamente un individuo particular, definimos efectos agregados para toda la población usando el valor esperado de $\tau_i$ de tres formas distintas:  el efecto promedio del tratamiento (ATE), el efecto promedio del tratamiento en los tratados (ATT) y el efecto promedio del tratamiento en los no tratados (ATU) como:

\begin{align}
\tau_{\text{ATE}} =& E[Y_i(1) - Y_i(0)] \\
\tau_{\text{ATU}} =& E[Y_i(1) - Y_i(0) | D_i = 0] \\
\tau_{\text{ATT}} =& E[Y_i(1) - Y_i(0) | D_i = 1]
\end{align}


Sabemos que, como no observamos para una persona ambos resultados potenciales, no podemos calcular directamente los efectos individuales. Por lo tanto, una comparación "ingenua" de los promedios de los tratados y no tratados no nos dará una estimación válida de los efectos de tratamiento:

\begin{align*}
\tau_{\text{ATE}} &\neq E[Y_i | D_i = 1] - E[Y_i | D_i = 0] \\
\end{align*}

\subsection{A3. }
\subsection{A4. Demostración consistencia estimador}
\subsection{A5. Análisis e interpretación ATE, ATT, ATU}
\subsection{A2. Representación matemática de mecanismo de asignación}

\colorbox{yellow}{esto no me convence}
Dado que, como explicamos previamente, el problema de inferencia causal deriva por que la asignación al tratamiento \textit{puede} es desconocida, podemos definir el mecanismo de asignación para poder comparar con mayor precisión las dos alternativas propuestas.

\begin{definition}
    Un mecanismo de asignación es una función de probabilidad que describe la probabilidad de asignación al tratamiento condicional a los resultados potenciales y a las covariables:
    $$\pi(D_i = 1 | Y_i(1), Y_i(0), X_i)$$
\end{definition}

\begin{definition}
    Decimos que un mecanismo es aleatorio (\textit{unconfounded}) si la asignación al tratamiento es independiente de los resultados potenciales y de las covariables. Es decir:
    $$\pi(D_i = 1 | Y_i(1), Y_i(0), X_i) = \pi(D_i = 1 | X_i)$$
\end{definition}


Para cada alternativa podemos definir matemáticamente el mecanismo de asignación así:

\begin{algorithm}
\caption{Alternativa 1}
\begin{algorithmic}
\FOR{cada investigador $i$}
    \FOR{cada tema de investigación $j$}
        \IF{investigador $i$ ha trabajado en tema $j$ antes}
            \STATE $\pi_{ij}(D_i = 1) = 0$ (\colorbox{yellow}{¿con subíndices o no?})
        \ELSE
            \STATE $\pi_{ij}(D_i = 1) = p_j$ (\colorbox{yellow}{tiene sentido que sea $p_j$?})
        \ENDIF
    \ENDFOR
\ENDFOR
\end{algorithmic}
\end{algorithm}

donde $p_j$ es la probabilidad de asignación al tema $j$ cuando el investigador es elegible.

\begin{algorithm}
\caption{Alternativa 2}
\begin{algorithmic}
\FOR{cada investigador $i$}
    \FOR{cada tema de investigación $j$}
        \IF{investigador $i$ ha trabajado en tema $j$ antes}
            \STATE $\pi_{ij}(D_i = 1 | X_i) = 0$
        \ELSE
            \STATE $\pi_{ij}(D_i = 1 | X_i) = f(\epsilon_i)$
        \ENDIF
    \ENDFOR
\ENDFOR
\end{algorithmic}
\end{algorithm}

donde $f(\epsilon_i)$ es una función que depende de las características no observables del investigador.

Antes de adentrarnos a explicar matemáticamente si la asignación es aleatoria o no, podemos ilustrar el problema con dos ejemplos.

\textbf{Ejemplo 1:} Supongamos que un investigador $i$ deliberadamente ha escogido en su carrera trabajar en temas que nadie trabaja, que resultan ser (\textit{\textbf{not} trending research topics}) por razones no observables $\epsilon_i$. Por lo tanto no ha trabajado en el tema $j$ que resulta ser \textit{trending research topic} antes. Por lo tanto, bajo la Alternativa 1, es seguro (altamente probable) que $i$ sea asignado al tratamiento (si no le asignan $j$, tal vez le asignen $j'$ que, por su historial de elecciónes, también será \textit{trending research topic}). EN este caso la asignación al tratamiento está "contaminada" por características sistemáticas de $i$, haciéndole 

En este caso, la probabilidad de asignación al tratamiento es $\pi_{ij}(D_i = 1, ) = 0$. Por lo tanto, el investigador no será asignado al tratamiento.

\colortext{yellow}{END EJEMPLO 1}

Ahora podemos definir la probabilidad de asignación al tratamiento para cada alternativa. Para la Alternativa 1, la probabilidad de asignación al tratamiento es:
\colorbox{yellow}{HELP!!}



\begin{enumerate}[label=A.\arabic*]
    \item Describa el problema de estimación usando el lenguaje de resultados potenciales.
    \begin{enumerate}[label=A.1.\alph*]
        \item Describa los resultados potenciales en contexto.
        \item Explique el problema de inferencia causal.
        \item Escriba e interprete el ATE, ATT y ATU.
    \end{enumerate}


    \item Representación matemática de los mecanismos de asignación.
    \begin{enumerate}[label=A.2.\alph*]
        \item ¿Es aleatoria la Alternativa 1? Justifique (150 palabras).
        \item ¿Es aleatoria la Alternativa 2? Justifique (150 palabras).
        \item Discuta problemas de inferencia causal si la asignación no es aleatoria (150 palabras).
    \end{enumerate}
    \item Demostración de la consistencia del estimador.
    \item Interpretación y análisis de los parámetros ATT, ATU, y ATE.
\end{enumerate}







%%%%%%%%%%%%%%%%%%%%%%%%%%%%%%%%%%%%%%%%%%%
% ▗▄▄▖  ▗▄▖ ▗▄▄▖▗▄▄▄▖▗▄▄▄▖    ▗▄▄▖ 
% ▐▌ ▐▌▐▌ ▐▌▐▌ ▐▌ █  ▐▌       ▐▌ ▐▌
% ▐▛▀▘ ▐▛▀▜▌▐▛▀▚▖ █  ▐▛▀▀▘    ▐▛▀▚▖
% ▐▌   ▐▌ ▐▌▐▌ ▐▌ █  ▐▙▄▄▖    ▐▙▄▞▘
%%%%%%%%%%%%%%%%%%%%%%%%%%%%%%%%%%%%%%%%%%%
                                 
                                
\section{Parte B: Teorema de Frisch-Waugh-Lovell}
\subsection{B1. Demostraciones relacionadas con proyecciones y residuales.}

Primero veamos que la matriz $\textbf{P} = \textbf{X}(\textbf{X}'\textbf{X})^{-1}\textbf{X}'$ es una proyección ortogonal. Para ello debemos ver si esta es idempotente y simétrica. En exte ejercicio utilizaremos las mayúsculas $X$ como matrices para facilitar la notación. 



\( P = X (X'X)^{-1} X' \), \quad \( X \in \mathbb{R}^{n \times k} \), \quad \( P \in \mathbb{R}^{n \times n} \)

\subsubsection*{Idempotencia:}
\begin{align*}
P P &= \left( X (X'X)^{-1} X' \right) \left( X (X'X)^{-1} X' \right) \\
    &= X (X'X)^{-1} (X'X) (X'X)^{-1} X' \\
    &= X (X'X)^{-1} X' \\
    &= P
\end{align*}

\subsubsection*{Simetría:}
\begin{align*}
P' &= \left( X (X'X)^{-1} X' \right)' \\
   &= \left(  \right) \\
   &= \left( X' \right)' \left( (X'X)^{-1} \right)' X' \\
   &= X \left( (X'X)^{-1} \right)' X' \\
   &= X \left( (X'X)' \right)^{-1} X' \\
   &= X \left( X'X \right)^{-1} X' \\
   &= P
\end{align*}


\subsection{B2. Regresión particionada: derivación y demostración.}


Sea $\boldsymbol{y} = \boldsymbol{X}\boldsymbol{\beta}$ el vector de valores predichos y recordemos que el estimador de MCO matricial es $\boldsymbol{\hat{\beta}} = (\boldsymbol{X}'\boldsymbol{X})^{-1}\boldsymbol{X}'\boldsymbol{y}$. 

\begin{align*}
    \boldsymbol{\hat{\beta}} &= (\boldsymbol{X}'\boldsymbol{X})^{-1}\boldsymbol{X}'\boldsymbol{y} \\
    % premultiplicar por X
    \boldsymbol{X}\boldsymbol{\hat{\beta}} &= \boldsymbol{X}(\boldsymbol{X}'\boldsymbol{X})^{-1}\boldsymbol{X}'\boldsymbol{y} \\
    % simplificar por valores predichos
    \boldsymbol{\hat{y}} &= \boldsymbol{X}(\boldsymbol{X}'\boldsymbol{X})^{-1}\boldsymbol{X}'\boldsymbol{\beta} \\
    % simplificar por matriz de proyección 
    \boldsymbol{\hat{y}} &= \mathbf{P} \boldsymbol{\beta} \\


\end{align*}


               
Veamos que $\boldsymbol{\hat{y}} = \boldsymbol{X}\boldsymbol{\hat{\beta}}$ es la proyección de $\boldsymbol{y}$ sobre el espacio generado por $\boldsymbol{X}$. Para ello, debemos demostrar que $\boldsymbol{y} - \boldsymbol{\hat{y}}$ es ortogonal a $\boldsymbol{X}$.

% ██████  ██████  
% ██   ██      ██ 
% ██████   █████  
% ██   ██      ██ 
% ██████  ██████  

Ahora veamos que la matriz $P_1 = X_1(X_1'X_1)^{-1}X_1'$ es una proyección ortogonal. Para ello debemos ver si esta es idempotente y simétrica

\textbf{Idempotencia:}
\begin{proof}
\begin{align*}
    P_1^2 &= X_1(X_1'X_1)^{-1}X_1'X_1(X_1'X_1)^{-1}X_1' \\
    &= X_1(X_1'X_1)^{-1}X_1' \\
    &= P_1
\end{align*}
\end{proof}

\textbf{Simetría:}
\begin{proof}
\begin{align*}
    P_1' &= \left( X_1(X_1'X_1)^{-1}X_1' \right)' \\
    &= X_1 \left( (X_1'X_1)^{-1} \right)' X_1' \\
    &= X_1 \left( (X_1'X_1)' \right)^{-1} X_1' \\
    &= X_1 (X_1'X_1)^{-1} X_1' \\
    &= P_1
\end{align*}
\end{proof}

Por otro lado, veamos que la matriz de aniquilación $M_1 = I - P_1$ es una proyección ortogonal. Para ello debemos ver si esta es idempotente y simétrica.

\textbf{Idempotencia:}
\begin{proof}
\begin{align*}
    M_1^2 &= (I - P_1)(I - P_1) \\
    &= I - P_1 - P_1 + P_1^2 \\
    &= I - P_1 - P_1 + P_1 \\
    &= I - P_1
    &= M_1
\end{align*}
\end{proof}

\textbf{Simetría:}
\begin{proof}
\begin{align*}
    M_1' &= (I - P_1)' \\
    &= I' - P_1' \\
    &= I - P_1 \\
    &= M_1
\end{align*}
\end{proof}

\bigskip


% ██████  ██████      █████  
% ██   ██      ██    ██   ██ 
% ██████   █████     ███████ 
% ██   ██      ██    ██   ██ 
% ██████  ██████  ██ ██   ██ 
                           
Ahora veamos que $M_1$ aniquila a $X_1$. Para ello, debemos demostrar que $M_1X_1 = \boldsymbol{0}$.
\begin{proof}
\begin{align*}
    M_1X_1 &= (I - P_1)X_1 \\
    &= X_1 - P_1X_1 \\
    &= X_1 - X_1(X_1'X_1)^{-1}X_1'X_1 \\
    &= X_1 - X_1 \\
    &= \boldsymbol{0}
\end{align*}
\end{proof}



% ██████  ██████     ██████  
% ██   ██      ██    ██   ██ 
% ██████   █████     ██████  
% ██   ██      ██    ██   ██ 
% ██████  ██████  ██ ██████  
                           
Por otro lado, veamos que $M_1\epsilon = \epsilon$. Esta propiedad nos dice que la matriz de aniquilación $M_1$ no afecta a $\epsilon$.
Para ello, debemos demostrar que $M_1\epsilon = \epsilon$.

\begin{proof}
    \begin{align}
        M_1\epsilon &= (I - P_1)\epsilon \\
        &= \epsilon - P_1\epsilon \qquad \colorbox{yellow}{(¿por qué?)}
        &= \epsilon 
    \end{align}
\end{proof}



% ██████  ██   ██ 
% ██   ██ ██   ██ 
% ██████  ███████ 
% ██   ██      ██ 
% ██████       ██ 


                



%%%%%%%%%%%%%%%%%%%%%%%%%%%%%%%%%%%%%%%%%%%
% ▗▄▄▖  ▗▄▖ ▗▄▄▖▗▄▄▄▖▗▄▄▄▖     ▗▄▄▖
% ▐▌ ▐▌▐▌ ▐▌▐▌ ▐▌ █  ▐▌       ▐▌   
% ▐▛▀▘ ▐▛▀▜▌▐▛▀▚▖ █  ▐▛▀▀▘    ▐▌   
% ▐▌   ▐▌ ▐▌▐▌ ▐▌ █  ▐▙▄▄▖    ▝▚▄▄▖
%%%%%%%%%%%%%%%%%%%%%%%%%%%%%%%%%%%%%%%%%%%

\section{Parte C: Aplicación Práctica — Conflicto en Darfur}
\begin{enumerate}[label=C.\arabic*]
    \item Discusión del problema de identificación causal.
    \item Uso del Teorema de Frisch-Waugh-Lovell para análisis de sesgo.
\end{enumerate}




\section{Parte D: Diseño Experimental en Punjab}
\subsection{Configuración del Experimento y Análisis}
\begin{enumerate}[label=D.\arabic*]
    \item Definición de asignación y resultados potenciales.
    \item Balance de muestra y especificación de efectos de tratamiento.
\end{enumerate}

%%%%%%%%%%%%%%%%%%%%%%%%%%%%%%%%%%%%%%%%%%%
% ▗▄▄▖  ▗▄▖ ▗▄▄▖▗▄▄▄▖▗▄▄▄▖    ▗▄▄▄ 
% ▐▌ ▐▌▐▌ ▐▌▐▌ ▐▌ █  ▐▌       ▐▌  █
% ▐▛▀▘ ▐▛▀▜▌▐▛▀▚▖ █  ▐▛▀▀▘    ▐▌  █
% ▐▌   ▐▌ ▐▌▐▌ ▐▌ █  ▐▙▄▄▖    ▐▙▄▄▀
%%%%%%%%%%%%%%%%%%%%%%%%%%%%%%%%%%%%%%%%%%%
\subsection{Estimación y Análisis de Regresión}
\begin{enumerate}[label=D.\arabic*]
    \item Estimación con controles y efectos fijos.
    \item Análisis del comportamiento oportunista y su impacto en los resultados.
\end{enumerate}

\section*{Conclusión}
\noindent En este documento se resolvieron los ejercicios planteados, abordando problemas de identificación causal, análisis de regresión y diseño experimental.

\end{document}
